\documentclass{article}
\usepackage{geometry}
\geometry{a4paper, margin=1in}
\usepackage{graphicx}
\usepackage[colorlinks=true, linkcolor=blue, citecolor=blue, urlcolor=blue]{hyperref}
\usepackage{listings}
\usepackage{xcolor}
\usepackage{amsmath}
\usepackage{enumitem}

\lstset{
    language=NASM,
    basicstyle=\ttfamily\footnotesize\selectfont, % use the selected monospaced font
    backgroundcolor=\color{white},
    keywordstyle=\color{blue},
    commentstyle=\color{gray},
    stringstyle=\color{red},
    numbers=left,
    numberstyle=\tiny\color{gray},
    stepnumber=1,
    numbersep=10pt,
    frame=single,
    breaklines=true,
    captionpos=b,
    tabsize=4
}

\title{Assignment 1 - Report}
\author{
    [Welby Seely] \\
    \texttt{[wseely@emich.edu]}
}
\date{\today}

\begin{document}

    \maketitle

    \section{References}\label{sec:references}
    \bibliographystyle{plainurl}
    \renewcommand{\refname}{} % Suppress the automatic bibliography title
    \bibliography{bibliography}

    \section{Design}\label{sec:design}
    Referenced The Standard ML Basis Library~\cite{sml_basis} and class notes~\cite{zhang_class_notes} while designing functions.
    \\
    Most functions were designed with an inner helper function, defined within the outer function's closure in order to encapsulate logic.
    For example:
    The inner function ``dl()'' is inaccessible from outside the primary function, ``delntch''.
    This keeps the design clean and the interface to the logic clear.

    \section{Requirements}\label{sec:difficulty}
    Some of these functions could likely be redesigned, reduced into a single function.
    For example, my original design of ``multin'' used a helper functions:
    \\
    \begin{lstlisting}[language=Python]
    \end{lstlisting}
    The function is reduced to just two or three lines!

    \section{Appendix 1: Source Code}\label{sec:appendix_1}
    \begin{lstlisting}

    \end{lstlisting}

\end{document}
